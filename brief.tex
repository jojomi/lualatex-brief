% !TeX program = lualatex
\documentclass[absender]{scrlttr2}

\newcommand{\topic}{Reklamation}
\usepackage{blindtext}
% Eigene Werte definieren, auf die im Brief zugegriffen werden kann
% \newcommand{\vertragsnummer}{45781258\xspace}

\begin{document}

\begin{letter}{
    Firma GmbH\\%
    Teststr. 23\\%
    89554 München 
}

\KOMAoptions{
    %fromphone
}

\setkomavar{subject}{Reklamation zur Rechnung 20202}

% ====== Geschäftszeichenzeile =========
\setkomavar{yourref}{}          % Ihr Zeichen
\setkomavar{yourmail}{}         % Ihr Schreiben vom
\setkomavar{myref}{}            % Unser Zeichen
\setkomavar{customer}{}         % Kundennummer
\setkomavar{invoice}{}          % Rechnungsnummer
\setkomavar{place}{} % Ort
\setkomavar{date}{\today}       % Datum
% =====================================

\opening{Sehr geehrte Damen und Herren,}

\Blindtext[4][1]

\closing{Mit freundlichen Grüßen,}

% ===== Postskriptum =====
\ps PS: \dots
% ========================

% ===== Anlage(n) =====
% \setkomavar*{enclseparator}{Anlage}
\encl{%
  Anlage 1\\
  Anlage 2%
}
% ===================

% ===== Verteiler =====
% \setkomavar*{ccseparator}{Kopie an}
\cc{%
  Verteiler 1\\
  Verteiler 2%
}
% =====================

\end{letter}


\end{document}